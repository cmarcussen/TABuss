With the constant increase in smartphone sales, integrated sensors and map navigation has now become available to the average user. This allows for mobile applications to use the user's context to provide more relevant information. An interesting use-case for such applications is a route information systems for buses.

This report covers the improvements made to Magnus Raaum's existing context-aware application for Android, made in 2011. The application uses BusTUC, a reasoning-system for bus routes in Trondheim. By combining user context with BusTUC reasoning and real-time data from AtB, the user-interaction is simplified, compared to a standard information application. 

The original Android application had the same basic functionality. We have built on, and extended this functionality, improved the design, and made the application ready for release to the public. Based on feedback from beta-testers, we have reason to believe that the improved application suits the needs of typical bus travellers.
